\chapter{注意など}
\label{chap:prelim}
\section{注意}
本原稿のキーワードは以下の通り:
\begin{enumerate}
\item 線形空間
\item 線形写像, 同型写像
\item Ker Img
\item 双対空間, Hom
\item 部分空間
\item (剰余空間と準同型定理)
\item 生成系
\item 一次独立性
\item 基底
\item 次元 (次元の一意性)
\item 基底の延長定理
\item 次元定理
\item 和と共通部分
\item 内部直和, 外部直和
\item 表現行列, 基底の変換行列
\end{enumerate}

行列や数ベクトルに関しては知っていたほうが理解は進むと思うが
前提としない.

有限次元のときを扱う.

内積については, 基本的に触らない.
体は一般でやる.
